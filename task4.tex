\section{Задание 4. Приложения определенного интеграла}
Вычислить работу, необходимую для извлечения деревянной прямоугольной балки, плавающей в воде, если длина балки 5 м, ширина 40 см , высота 20 см, а ее удельный вес равен 0,8.\\
Удельный вес \(\gamma = \frac{P}{V}\), где P - вес балки, V - объём.\\
Поскольку балка плавает в воде вес балки равен весу воды, вытесняемой подводной частью балки.\\
т.е. \(0.8 \cdot (0.4 \cdot 5 \cdot 0.2) = 1 \cdot (0.4 \cdot 5 \cdot H)\), откуда \(H = 0.16\), , поскольку удельный вес воды - 1\\значит под водой находится 16см балки.\\
Чтобы достать балку из воды необходимо понять её на 16см = 0.16м.\\
F(h) - сила, которую надо приложить, чтобы поднять балку на высоту h.\\
\(F(h) = \gamma \cdot (0.4 \cdot 5 \cdot 0.2) - 1 \cdot (0.16-h) \cdot 0.4 \cdot 5\cdot \)
Таким образом, работа необходимую для извлечения балки \(A = \int_0^{0.1}(F(h))\)
\begin{flalign*}
&\int_0^{0.16} (0.32 - 2 (0.16 - h))dh = \int_0^{0.16} 2h dh = 0.0256 \text{Дж}&
\end{flalign*}