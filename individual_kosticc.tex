\section{Задание 4. Приложение Рядов. Лучинкин Константин Вариант 23}
\subsection{Задание 1.}
\begin{flalign*}
& f(x) = e^x &
\end{flalign*}
Требуется посчитать \(f(-\frac{9}{10}) = \frac{1}{\sqrt[10]{e^9}}\)\\
Разложим \(f(x)\) в ряд Тейлора:
\begin{flalign*}
&f(x) = \sum_{n=0}^\infty \frac{x^n}{n!}&
\end{flalign*}
Рассмотрим частичные суммы рядя \(S_{(N,x)} = \sum_{n=0}^N \frac{x^n}{n!}\):\newline
\begin{flalign*}
& S_{(1, x)} = 1 + x;\quad S_{(1, -0.9)} = 0.1 &\\
& S_{(2, x)} = 1 + x + \frac{x^2}{2};\quad S_{(2, -0.9)} = 0.505 &\\
& S_{(3, x)} = 1 + x + \frac{x^2}{2} + \frac{x^3}{6};\quad S_{(3, -0.9)} = 0.3835 &\\
&...&\\
& S_{(7, x)} = \sum_{n=0}^7 \frac{x^7}{7!};\quad S_{(7, -0.9)} \approx 0.40655 &\\
& S_{(8, x)} = \sum_{n=0}^8 \frac{x^8}{8!};\quad S_{(8, -0.9)} \approx 0.40657 &
\end{flalign*}
Таким образом \(f(x) \approx 0.4065\) (с точностью до 0.0001)
\subsection{Задание 2}
\begin{flalign*}
    &\int_0^1 \frac{x^2dx}{\sqrt[4]{16 + x^4}}&\\
    &\text{Разложим }f(x) \text{ в ряд:}&\\
    &f(x) = \frac{x^2}{\sqrt[4]{16 + x^4}} = \frac{x^2}{2\sqrt[4]{\frac{x^4}{16} + 1}} = \frac{x^2}{2}\Big(\frac{x^4}{16}+1\Big)^{-\frac{1}{4}} = \frac{x^2}{2}\Big(1 + \sum_{n=0}^\infty \frac{\prod_{i=1}^{n-1}\Big(-\frac{1}{4}-i\Big)}{n!}\Big(\frac{x^4}{16}\Big)^n\Big)&\\
    &\int_0^1 \Big(\frac{x^2}{2}\Big)dx \approx 0.1666&\\
    &\int_0^1 \Big(\frac{x^2}{2} - \frac{x^6}{128}\Big)dx \approx 0.1655 &\\
    &\int_0^1 \Big(\frac{x^2}{2} - \frac{x^6}{128} + \frac{5x^{10}}{16384}\Big)dx \approx 0.1655 
\end{flalign*}
Таким образом \(\int_0^1 \frac{x^2dx}{\sqrt[4]{16 + x^4}} \approx 0.1655\) (с точностью до 0.0001)
\subsection{Задание 3}
Найти в виде степенного ряда решение дифференциального уравнения,
удовлетворяющего заданным начальным условиям. Ограничиться четырьмя
членами ряда.
\begin{flalign*}
    &y'' = e^{y-2}\ln y' + x &\\
    &y(3) = 2 &\\
    &y'(3) = 1&
\end{flalign*}
Запишем формулу Тейлора для \(x=3\):
\begin{flalign*}
    &f(x) = y(3) + y'(3)(x-3) + y''(3)(x-3)^2 + y'''(3)(x_3)^3&\\
    &y''(3) = e^{y(3)-2}\ln y'(3) + 3 = 3&\\
    &y'''(3) = e^{y(3)-2}(1) + 1 = 2&\\
    &f(x) = 2 + (x-3) + 3(x-3)^2 + 2(x-3)^3 + r\quad \quad \quad \quad r \text{-остаточный член}&
\end{flalign*}