\section{Задание 3. Несобственный интеграл}
Исследуем сходимость \( \int_{1}^{\infty} \frac{dx}{(x^\alpha) 1)arctg(x)} \)
\begin{flalign*}
    &\lim_{x \to \infty} \frac{1}{(x^\alpha + 1)arctg(x)(\frac{1}{x^\alpha})} < lim_{x \to \infty} \frac{1}{(x^\alpha + 1)\frac{\pi}{2}(\frac{1}{x^\alpha})},&\\
\end{flalign*}
Поскольку \(\arctg(x) < \frac{\pi}{2}\)\\
\begin{flalign*}
    &\lim_{x \to \infty} \frac{1}{(x^\alpha + 1)\frac{\pi}{2}(\frac{1}{x^\alpha})} = \lim_{x \to \infty} \frac{x^\alpha}{\frac{\pi}{2}x^\alpha + \frac{\pi}{2}} = \frac{2}{\pi}
    &\\
\end{flalign*}
Тогда по предельному признаку сравнения \( \int_{1}^{\infty} \frac{dx}{(x^\alpha) + 1)\arctg(x)}\)  сходится или расходится одновременно с \(\int_{1}^{\infty} \frac{dx}{x^\alpha}\). Таким образом \(\int_{1}^{\infty} \frac{dx}{(x^\alpha) + 1)\arctg(x)} \) расходится при \(\alpha \leq 1\) и сходится при \(\alpha > 1\)
