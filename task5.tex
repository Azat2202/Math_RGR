\section{Задание 5. Приближенные вычисления определенного интеграла}
\subsection*{Задание}
Найти приближенное значение интеграла $ \int_0^{10}{e^{-x}	dx} = 0,999955 $ методами прямоугольников, трапеций, парабол , Боде при h=1. Сделать анализ полученных результатов.
\subsection*{Метод прямоугольников}
Отрезок $ [a;b] $ разбивается на равные промежутки длины $ h $, $ t $ - положение точки в разбиении 
\begin{align*}
	&h = \frac{b - a}{n} \\
	&\epsilon_i = x_{i - 1} + h * t \\
	&\int_{a}^{b} f(x)dx \approx h * \sum_{i = 1}^{n} f\left(x_{i-1} + h*t \right)
\end{align*}
Требуемая точность достигается при $ n = 10\_000 \text{ и } t = 0.5$
\begin{lstlisting}[language=Python]
def rectangle_method(f, a, b, n, t):
	h = (b - a) / n
	result = sum([f(a + (i - t) * h) for i in range(1, n + 1)]) * h
	return result
\end{lstlisting}
\subsection*{Метод трапеций}
\begin{align*}
	&x_i = a + ih \\
	&h = \frac{b - a}{n} \\
	&\int_a^b f(x) dx \approx h * \left(\frac{f_0 + f_n}{2} + \sum_{i = 1}^{n-1} f_i\right)
\end{align*}
Требуемая точность достигается при $ n = 10\_000 $
\begin{lstlisting}[language=Python]
def trapezoid_method(f, a, b, n):
	h = (b - a) / n
	result = 0.5 * f(a) 
		+ sum([f(a + i*h) for i in range(1, n)]) 
		+ 0.5 * f(b)
	result *= h
	return result
\end{lstlisting}
\subsection*{Метод парабол}
\begin{align*}
	&x_j = a + jh \\
	&h = \frac{b - a}{N} \\
	&\int_a^b f(x) dx \approx \frac{h}{3} \left(f(x_0)  + 2\sum_{j=1}^{N/2-1}f(x_{2j}) + 4\sum_{j=1}^{N/2}f(x_{2j-1}) + f(x_N)\right)
\end{align*}
Требуемая точность достигается при $ n = 90 $
\begin{lstlisting}[language=Python]
def parabola_method(f, a, b,n):
	h = (b - a) / n
	return (h / 3) * (sum([(f(a + h * (i - 1)) 
		+ 4 * f(a + h * i) 
		+ f(a + h * (i + 1))) for i in range(1, n, 2)]))
\end{lstlisting}
\newpage
\subsection*{Метод Боде}
Имея N разбиений
\begin{dmath*}
	\int_{x_0}^{x_N} f(x) dx \approx \frac{2h}{45} \left(7(f(x_0) + f(x_N)) + 32\left(\sum_{i\in1, 3, 5, \dots , N - 1} f(x_i)\right) + 12\left(\sum_{i\in2, 6, 10, \dots , N - 2} f(x_i)\right) + 14\left(\sum_{i\in4, 8, 12, \dots , N - 4} f(x_i)\right)\right)
\end{dmath*}
Дает точность в 2 знака после запятой
\begin{lstlisting}[language=Python]
def bode_method(f, a, b, h):
	splitting = [(a + h*i) for i in range(0, int((b - a) / h) + 1)]
	n = len(splitting)
	return (2 * h / 45 ) * (7 * (f(splitting[0]) + f(splitting[-1])) 
		+ 32 * (sum([f(splitting[i]) for i in range(1, n, 2)]))
		+ 12 * (sum([f(splitting[i]) for i in range(2, n - 1, 4)])) 
		+ 14 * (sum([f(splitting[i]) for i in range(4, n - 3, 4)])))
\end{lstlisting}
\subsection*{Заключение}
\begin{tabular}{|c|c|c|c|c|}
	\hline
	&  Метод прямоугольникв & Метод трапеций  & Метод парабол & Метод Боде \\
	\hline
	N & 10\_000 & 10\_000 & 90 & При N = 10 точность не достигнута \\
	\hline
\end{tabular}

Самый удобный, быстрый и точный метод - парабол. 
\url{https://colab.research.google.com/drive/1oaMYtnWpOiJltNrdXcKLQt34urXhq8pk?usp=sharing}